%% Generated by Sphinx.
\def\sphinxdocclass{report}
\documentclass[letterpaper,10pt,english]{sphinxmanual}
\ifdefined\pdfpxdimen
   \let\sphinxpxdimen\pdfpxdimen\else\newdimen\sphinxpxdimen
\fi \sphinxpxdimen=.75bp\relax

\PassOptionsToPackage{warn}{textcomp}
\usepackage[utf8]{inputenc}
\ifdefined\DeclareUnicodeCharacter
% support both utf8 and utf8x syntaxes
  \ifdefined\DeclareUnicodeCharacterAsOptional
    \def\sphinxDUC#1{\DeclareUnicodeCharacter{"#1}}
  \else
    \let\sphinxDUC\DeclareUnicodeCharacter
  \fi
  \sphinxDUC{00A0}{\nobreakspace}
  \sphinxDUC{2500}{\sphinxunichar{2500}}
  \sphinxDUC{2502}{\sphinxunichar{2502}}
  \sphinxDUC{2514}{\sphinxunichar{2514}}
  \sphinxDUC{251C}{\sphinxunichar{251C}}
  \sphinxDUC{2572}{\textbackslash}
\fi
\usepackage{cmap}
\usepackage[T1]{fontenc}
\usepackage{amsmath,amssymb,amstext}
\usepackage{babel}



\usepackage{times}
\expandafter\ifx\csname T@LGR\endcsname\relax
\else
% LGR was declared as font encoding
  \substitutefont{LGR}{\rmdefault}{cmr}
  \substitutefont{LGR}{\sfdefault}{cmss}
  \substitutefont{LGR}{\ttdefault}{cmtt}
\fi
\expandafter\ifx\csname T@X2\endcsname\relax
  \expandafter\ifx\csname T@T2A\endcsname\relax
  \else
  % T2A was declared as font encoding
    \substitutefont{T2A}{\rmdefault}{cmr}
    \substitutefont{T2A}{\sfdefault}{cmss}
    \substitutefont{T2A}{\ttdefault}{cmtt}
  \fi
\else
% X2 was declared as font encoding
  \substitutefont{X2}{\rmdefault}{cmr}
  \substitutefont{X2}{\sfdefault}{cmss}
  \substitutefont{X2}{\ttdefault}{cmtt}
\fi


\usepackage[Bjarne]{fncychap}
\usepackage{sphinx}

\fvset{fontsize=\small}
\usepackage{geometry}

% Include hyperref last.
\usepackage{hyperref}
% Fix anchor placement for figures with captions.
\usepackage{hypcap}% it must be loaded after hyperref.
% Set up styles of URL: it should be placed after hyperref.
\urlstyle{same}
\addto\captionsenglish{\renewcommand{\contentsname}{Contents:}}

\usepackage{sphinxmessages}
\setcounter{tocdepth}{1}



\title{cloudmesh-analytics}
\date{Nov 05, 2019}
\release{}
\author{Qiwei Liu, Yanting Wan}
\newcommand{\sphinxlogo}{\vbox{}}
\renewcommand{\releasename}{}
\makeindex
\begin{document}

\pagestyle{empty}
\sphinxmaketitle
\pagestyle{plain}
\sphinxtableofcontents
\pagestyle{normal}
\phantomsection\label{\detokenize{index::doc}}



\chapter{cloudmesh package}
\label{\detokenize{cloudmesh:cloudmesh-package}}\label{\detokenize{cloudmesh::doc}}

\section{Subpackages}
\label{\detokenize{cloudmesh:subpackages}}

\subsection{cloudmesh.analytics package}
\label{\detokenize{cloudmesh.analytics:cloudmesh-analytics-package}}\label{\detokenize{cloudmesh.analytics::doc}}

\subsubsection{Subpackages}
\label{\detokenize{cloudmesh.analytics:subpackages}}

\paragraph{cloudmesh.analytics.api package}
\label{\detokenize{cloudmesh.analytics.api:cloudmesh-analytics-api-package}}\label{\detokenize{cloudmesh.analytics.api::doc}}

\subparagraph{Submodules}
\label{\detokenize{cloudmesh.analytics.api:submodules}}

\subparagraph{cloudmesh.analytics.api.manager module}
\label{\detokenize{cloudmesh.analytics.api:module-cloudmesh.analytics.api.manager}}\label{\detokenize{cloudmesh.analytics.api:cloudmesh-analytics-api-manager-module}}\index{cloudmesh.analytics.api.manager (module)@\spxentry{cloudmesh.analytics.api.manager}\spxextra{module}}\index{Manager (class in cloudmesh.analytics.api.manager)@\spxentry{Manager}\spxextra{class in cloudmesh.analytics.api.manager}}

\begin{fulllineitems}
\phantomsection\label{\detokenize{cloudmesh.analytics.api:cloudmesh.analytics.api.manager.Manager}}\pysigline{\sphinxbfcode{\sphinxupquote{class }}\sphinxcode{\sphinxupquote{cloudmesh.analytics.api.manager.}}\sphinxbfcode{\sphinxupquote{Manager}}}
Bases: \sphinxcode{\sphinxupquote{object}}
\index{list() (cloudmesh.analytics.api.manager.Manager method)@\spxentry{list()}\spxextra{cloudmesh.analytics.api.manager.Manager method}}

\begin{fulllineitems}
\phantomsection\label{\detokenize{cloudmesh.analytics.api:cloudmesh.analytics.api.manager.Manager.list}}\pysiglinewithargsret{\sphinxbfcode{\sphinxupquote{list}}}{\emph{parameter}}{}
\end{fulllineitems}


\end{fulllineitems}



\subparagraph{Module contents}
\label{\detokenize{cloudmesh.analytics.api:module-cloudmesh.analytics.api}}\label{\detokenize{cloudmesh.analytics.api:module-contents}}\index{cloudmesh.analytics.api (module)@\spxentry{cloudmesh.analytics.api}\spxextra{module}}

\paragraph{cloudmesh.analytics.command package}
\label{\detokenize{cloudmesh.analytics.command:cloudmesh-analytics-command-package}}\label{\detokenize{cloudmesh.analytics.command::doc}}

\subparagraph{Submodules}
\label{\detokenize{cloudmesh.analytics.command:submodules}}

\subparagraph{cloudmesh.analytics.command.analytics module}
\label{\detokenize{cloudmesh.analytics.command:module-cloudmesh.analytics.command.analytics}}\label{\detokenize{cloudmesh.analytics.command:cloudmesh-analytics-command-analytics-module}}\index{cloudmesh.analytics.command.analytics (module)@\spxentry{cloudmesh.analytics.command.analytics}\spxextra{module}}\index{AnalyticsCommand (class in cloudmesh.analytics.command.analytics)@\spxentry{AnalyticsCommand}\spxextra{class in cloudmesh.analytics.command.analytics}}

\begin{fulllineitems}
\phantomsection\label{\detokenize{cloudmesh.analytics.command:cloudmesh.analytics.command.analytics.AnalyticsCommand}}\pysigline{\sphinxbfcode{\sphinxupquote{class }}\sphinxcode{\sphinxupquote{cloudmesh.analytics.command.analytics.}}\sphinxbfcode{\sphinxupquote{AnalyticsCommand}}}
Bases: \sphinxcode{\sphinxupquote{cloudmesh.shell.command.PluginCommand}}
\index{do\_analytics() (cloudmesh.analytics.command.analytics.AnalyticsCommand method)@\spxentry{do\_analytics()}\spxextra{cloudmesh.analytics.command.analytics.AnalyticsCommand method}}

\begin{fulllineitems}
\phantomsection\label{\detokenize{cloudmesh.analytics.command:cloudmesh.analytics.command.analytics.AnalyticsCommand.do_analytics}}\pysiglinewithargsret{\sphinxbfcode{\sphinxupquote{do\_analytics}}}{\emph{args}}{}~
\begin{sphinxVerbatim}[commandchars=\\\{\}]
\PYG{n}{Usage}\PYG{p}{:}
      \PYG{n}{analytics} \PYG{n}{server} \PYG{n}{start} \PYG{p}{[}\PYG{o}{\PYGZhy{}}\PYG{o}{\PYGZhy{}}\PYG{n}{cloud}\PYG{o}{=}\PYG{n}{CLOUD}\PYG{p}{]}
      \PYG{n}{analytics} \PYG{n}{server} \PYG{n}{stop} \PYG{p}{[}\PYG{o}{\PYGZhy{}}\PYG{o}{\PYGZhy{}}\PYG{n}{cloud}\PYG{o}{=}\PYG{n}{CLOUD}\PYG{p}{]}

\PYG{n}{This} \PYG{n}{command} \PYG{n}{manages} \PYG{n}{the} \PYG{n}{cloudmesh} \PYG{n}{analytics} \PYG{n}{server} \PYG{n}{on} \PYG{n}{the} \PYG{n}{given} \PYG{n}{cloud}\PYG{o}{.}
\PYG{n}{If} \PYG{n}{the} \PYG{n}{cloud} \PYG{o+ow}{is} \PYG{o+ow}{not} \PYG{n}{spified} \PYG{n}{it} \PYG{o+ow}{is} \PYG{n}{run} \PYG{n}{on} \PYG{n}{localhost}

\PYG{n}{Options}\PYG{p}{:}
    \PYG{o}{\PYGZhy{}}\PYG{o}{\PYGZhy{}}\PYG{n}{clout}\PYG{o}{=}\PYG{n}{CLOUD}  \PYG{n}{The} \PYG{n}{name} \PYG{n}{of} \PYG{n}{the} \PYG{n}{cloud} \PYG{k}{as} \PYG{n}{specified} \PYG{o+ow}{in} \PYG{n}{the}
                   \PYG{n}{cloudmesh}\PYG{o}{.}\PYG{n}{yaml} \PYG{n}{file}
\end{sphinxVerbatim}

\end{fulllineitems}


\end{fulllineitems}



\subparagraph{Module contents}
\label{\detokenize{cloudmesh.analytics.command:module-cloudmesh.analytics.command}}\label{\detokenize{cloudmesh.analytics.command:module-contents}}\index{cloudmesh.analytics.command (module)@\spxentry{cloudmesh.analytics.command}\spxextra{module}}

\paragraph{cloudmesh.analytics.server package}
\label{\detokenize{cloudmesh.analytics.server:cloudmesh-analytics-server-package}}\label{\detokenize{cloudmesh.analytics.server::doc}}

\subparagraph{Submodules}
\label{\detokenize{cloudmesh.analytics.server:submodules}}

\subparagraph{cloudmesh.analytics.server.db module}
\label{\detokenize{cloudmesh.analytics.server:module-cloudmesh.analytics.server.db}}\label{\detokenize{cloudmesh.analytics.server:cloudmesh-analytics-server-db-module}}\index{cloudmesh.analytics.server.db (module)@\spxentry{cloudmesh.analytics.server.db}\spxextra{module}}\index{close\_db() (in module cloudmesh.analytics.server.db)@\spxentry{close\_db()}\spxextra{in module cloudmesh.analytics.server.db}}

\begin{fulllineitems}
\phantomsection\label{\detokenize{cloudmesh.analytics.server:cloudmesh.analytics.server.db.close_db}}\pysiglinewithargsret{\sphinxcode{\sphinxupquote{cloudmesh.analytics.server.db.}}\sphinxbfcode{\sphinxupquote{close\_db}}}{\emph{e=None}}{}
\end{fulllineitems}

\index{get\_db() (in module cloudmesh.analytics.server.db)@\spxentry{get\_db()}\spxextra{in module cloudmesh.analytics.server.db}}

\begin{fulllineitems}
\phantomsection\label{\detokenize{cloudmesh.analytics.server:cloudmesh.analytics.server.db.get_db}}\pysiglinewithargsret{\sphinxcode{\sphinxupquote{cloudmesh.analytics.server.db.}}\sphinxbfcode{\sphinxupquote{get\_db}}}{}{}
\end{fulllineitems}

\index{init\_app() (in module cloudmesh.analytics.server.db)@\spxentry{init\_app()}\spxextra{in module cloudmesh.analytics.server.db}}

\begin{fulllineitems}
\phantomsection\label{\detokenize{cloudmesh.analytics.server:cloudmesh.analytics.server.db.init_app}}\pysiglinewithargsret{\sphinxcode{\sphinxupquote{cloudmesh.analytics.server.db.}}\sphinxbfcode{\sphinxupquote{init\_app}}}{\emph{app}}{}
\end{fulllineitems}

\index{init\_db() (in module cloudmesh.analytics.server.db)@\spxentry{init\_db()}\spxextra{in module cloudmesh.analytics.server.db}}

\begin{fulllineitems}
\phantomsection\label{\detokenize{cloudmesh.analytics.server:cloudmesh.analytics.server.db.init_db}}\pysiglinewithargsret{\sphinxcode{\sphinxupquote{cloudmesh.analytics.server.db.}}\sphinxbfcode{\sphinxupquote{init\_db}}}{}{}
\end{fulllineitems}



\subparagraph{cloudmesh.analytics.server.server module}
\label{\detokenize{cloudmesh.analytics.server:module-cloudmesh.analytics.server.server}}\label{\detokenize{cloudmesh.analytics.server:cloudmesh-analytics-server-server-module}}\index{cloudmesh.analytics.server.server (module)@\spxentry{cloudmesh.analytics.server.server}\spxextra{module}}
To create a flask app
\begin{description}
\item[{The method definition to create a flask app by call ing the create\_app function}] \leavevmode\begin{description}
\item[{Example:}] \leavevmode
create\_app(test\_config)

\end{description}

\end{description}
\index{create\_app() (in module cloudmesh.analytics.server.server)@\spxentry{create\_app()}\spxextra{in module cloudmesh.analytics.server.server}}

\begin{fulllineitems}
\phantomsection\label{\detokenize{cloudmesh.analytics.server:cloudmesh.analytics.server.server.create_app}}\pysiglinewithargsret{\sphinxcode{\sphinxupquote{cloudmesh.analytics.server.server.}}\sphinxbfcode{\sphinxupquote{create\_app}}}{\emph{config=None}}{}
To create a flask app
\begin{quote}\begin{description}
\item[{Parameters}] \leavevmode
\sphinxstyleliteralstrong{\sphinxupquote{config}} \textendash{} A dictionary contains the configurations for the flask app

\item[{Returns}] \leavevmode
A flask app object

\end{description}\end{quote}

\end{fulllineitems}



\subparagraph{Module contents}
\label{\detokenize{cloudmesh.analytics.server:module-cloudmesh.analytics.server}}\label{\detokenize{cloudmesh.analytics.server:module-contents}}\index{cloudmesh.analytics.server (module)@\spxentry{cloudmesh.analytics.server}\spxextra{module}}

\subsubsection{Submodules}
\label{\detokenize{cloudmesh.analytics:submodules}}

\subsubsection{cloudmesh.analytics.analytics module}
\label{\detokenize{cloudmesh.analytics:module-cloudmesh.analytics.analytics}}\label{\detokenize{cloudmesh.analytics:cloudmesh-analytics-analytics-module}}\index{cloudmesh.analytics.analytics (module)@\spxentry{cloudmesh.analytics.analytics}\spxextra{module}}
The analytic functions
The module include analytic functions, and are also the endpoints of the flask app. Those functions are referred by the
OpenAPI specification by operationIDs
\index{kmeans\_fit() (in module cloudmesh.analytics.analytics)@\spxentry{kmeans\_fit()}\spxextra{in module cloudmesh.analytics.analytics}}

\begin{fulllineitems}
\phantomsection\label{\detokenize{cloudmesh.analytics:cloudmesh.analytics.analytics.kmeans_fit}}\pysiglinewithargsret{\sphinxcode{\sphinxupquote{cloudmesh.analytics.analytics.}}\sphinxbfcode{\sphinxupquote{kmeans\_fit}}}{\emph{file\_name}, \emph{body}}{}
\end{fulllineitems}

\index{linear\_regression() (in module cloudmesh.analytics.analytics)@\spxentry{linear\_regression()}\spxextra{in module cloudmesh.analytics.analytics}}

\begin{fulllineitems}
\phantomsection\label{\detokenize{cloudmesh.analytics:cloudmesh.analytics.analytics.linear_regression}}\pysiglinewithargsret{\sphinxcode{\sphinxupquote{cloudmesh.analytics.analytics.}}\sphinxbfcode{\sphinxupquote{linear\_regression}}}{\emph{file\_name}, \emph{body}}{}
Linear regression.
\begin{quote}\begin{description}
\item[{Parameters}] \leavevmode\begin{itemize}
\item {} 
\sphinxstyleliteralstrong{\sphinxupquote{file\_name}} (\sphinxstyleliteralemphasis{\sphinxupquote{str}}) \textendash{} The file name that has the input data.

\item {} 
\sphinxstyleliteralstrong{\sphinxupquote{body}} (\sphinxstyleliteralemphasis{\sphinxupquote{dict}}) \textendash{} The request body, which is a dictionary mapped by the connexion.

\end{itemize}

\item[{Returns}] \leavevmode
Return an json objects.

\end{description}\end{quote}

\begin{sphinxadmonition}{warning}{Warning:}
The input format should be specified
\end{sphinxadmonition}

\end{fulllineitems}

\index{pca() (in module cloudmesh.analytics.analytics)@\spxentry{pca()}\spxextra{in module cloudmesh.analytics.analytics}}

\begin{fulllineitems}
\phantomsection\label{\detokenize{cloudmesh.analytics:cloudmesh.analytics.analytics.pca}}\pysiglinewithargsret{\sphinxcode{\sphinxupquote{cloudmesh.analytics.analytics.}}\sphinxbfcode{\sphinxupquote{pca}}}{}{}
\end{fulllineitems}



\subsubsection{cloudmesh.analytics.file module}
\label{\detokenize{cloudmesh.analytics:module-cloudmesh.analytics.file}}\label{\detokenize{cloudmesh.analytics:cloudmesh-analytics-file-module}}\index{cloudmesh.analytics.file (module)@\spxentry{cloudmesh.analytics.file}\spxextra{module}}
File operations
The module include file operations
\index{list() (in module cloudmesh.analytics.file)@\spxentry{list()}\spxextra{in module cloudmesh.analytics.file}}

\begin{fulllineitems}
\phantomsection\label{\detokenize{cloudmesh.analytics:cloudmesh.analytics.file.list}}\pysiglinewithargsret{\sphinxcode{\sphinxupquote{cloudmesh.analytics.file.}}\sphinxbfcode{\sphinxupquote{list}}}{}{}
List all uploaded files

\end{fulllineitems}

\index{read() (in module cloudmesh.analytics.file)@\spxentry{read()}\spxextra{in module cloudmesh.analytics.file}}

\begin{fulllineitems}
\phantomsection\label{\detokenize{cloudmesh.analytics:cloudmesh.analytics.file.read}}\pysiglinewithargsret{\sphinxcode{\sphinxupquote{cloudmesh.analytics.file.}}\sphinxbfcode{\sphinxupquote{read}}}{\emph{file\_name}}{}
Read files given a file name.
\begin{quote}\begin{description}
\item[{Parameters}] \leavevmode
\sphinxstyleliteralstrong{\sphinxupquote{file\_names}} \textendash{} The input data source.

\item[{Returns}] \leavevmode
Return a json response.

\end{description}\end{quote}

\end{fulllineitems}

\index{upload() (in module cloudmesh.analytics.file)@\spxentry{upload()}\spxextra{in module cloudmesh.analytics.file}}

\begin{fulllineitems}
\phantomsection\label{\detokenize{cloudmesh.analytics:cloudmesh.analytics.file.upload}}\pysiglinewithargsret{\sphinxcode{\sphinxupquote{cloudmesh.analytics.file.}}\sphinxbfcode{\sphinxupquote{upload}}}{\emph{file=None}}{}
Upload files to the server
:param file: A file stream
\begin{quote}\begin{description}
\item[{Returns}] \leavevmode
Return the file name if it success

\end{description}\end{quote}

\begin{sphinxadmonition}{attention}{Attention:}
Only support the csv format now.
\end{sphinxadmonition}
\begin{quote}\begin{description}
\item[{Raises}] \leavevmode
\sphinxstyleliteralstrong{\sphinxupquote{Raise an error message if the file format is not supported}} \textendash{} 

\end{description}\end{quote}

\end{fulllineitems}



\subsubsection{cloudmesh.analytics.file\_helpers module}
\label{\detokenize{cloudmesh.analytics:module-cloudmesh.analytics.file_helpers}}\label{\detokenize{cloudmesh.analytics:cloudmesh-analytics-file-helpers-module}}\index{cloudmesh.analytics.file\_helpers (module)@\spxentry{cloudmesh.analytics.file\_helpers}\spxextra{module}}
The helper function isolates non-endpoint function from the file module
\index{allowed() (in module cloudmesh.analytics.file\_helpers)@\spxentry{allowed()}\spxextra{in module cloudmesh.analytics.file\_helpers}}

\begin{fulllineitems}
\phantomsection\label{\detokenize{cloudmesh.analytics:cloudmesh.analytics.file_helpers.allowed}}\pysiglinewithargsret{\sphinxcode{\sphinxupquote{cloudmesh.analytics.file\_helpers.}}\sphinxbfcode{\sphinxupquote{allowed}}}{\emph{file\_name}, \emph{allowed\_extentions}}{}
The allowed file extensions
\begin{quote}\begin{description}
\item[{Parameters}] \leavevmode\begin{itemize}
\item {} 
\sphinxstyleliteralstrong{\sphinxupquote{file\_name}} \textendash{} The file name to check

\item {} 
\sphinxstyleliteralstrong{\sphinxupquote{allowed\_extensions}} \textendash{} The allowed file extensions

\end{itemize}

\item[{Returns}] \leavevmode
Return true or false

\end{description}\end{quote}

\end{fulllineitems}

\index{read\_csv() (in module cloudmesh.analytics.file\_helpers)@\spxentry{read\_csv()}\spxextra{in module cloudmesh.analytics.file\_helpers}}

\begin{fulllineitems}
\phantomsection\label{\detokenize{cloudmesh.analytics:cloudmesh.analytics.file_helpers.read_csv}}\pysiglinewithargsret{\sphinxcode{\sphinxupquote{cloudmesh.analytics.file\_helpers.}}\sphinxbfcode{\sphinxupquote{read\_csv}}}{\emph{file\_name}}{}
Read csv using panda. The source path is relative and set when initializing flask app.
\begin{quote}\begin{description}
\item[{Parameters}] \leavevmode
\sphinxstyleliteralstrong{\sphinxupquote{file\_name}} \textendash{} The file name to read

\item[{Returns}] \leavevmode
A numpy array

\end{description}\end{quote}

\end{fulllineitems}

\index{save() (in module cloudmesh.analytics.file\_helpers)@\spxentry{save()}\spxextra{in module cloudmesh.analytics.file\_helpers}}

\begin{fulllineitems}
\phantomsection\label{\detokenize{cloudmesh.analytics:cloudmesh.analytics.file_helpers.save}}\pysiglinewithargsret{\sphinxcode{\sphinxupquote{cloudmesh.analytics.file\_helpers.}}\sphinxbfcode{\sphinxupquote{save}}}{\emph{file}}{}
Save file after securing the file name
\begin{quote}\begin{description}
\item[{Parameters}] \leavevmode
\sphinxstyleliteralstrong{\sphinxupquote{file}} \textendash{} the input data source

\item[{Returns}] \leavevmode
Return a json response

\end{description}\end{quote}

\end{fulllineitems}



\subsubsection{Module contents}
\label{\detokenize{cloudmesh.analytics:module-cloudmesh.analytics}}\label{\detokenize{cloudmesh.analytics:module-contents}}\index{cloudmesh.analytics (module)@\spxentry{cloudmesh.analytics}\spxextra{module}}

\section{Module contents}
\label{\detokenize{cloudmesh:module-cloudmesh}}\label{\detokenize{cloudmesh:module-contents}}\index{cloudmesh (module)@\spxentry{cloudmesh}\spxextra{module}}

\chapter{tests package}
\label{\detokenize{tests:tests-package}}\label{\detokenize{tests::doc}}

\section{Submodules}
\label{\detokenize{tests:submodules}}

\section{tests.conftest module}
\label{\detokenize{tests:module-tests.conftest}}\label{\detokenize{tests:tests-conftest-module}}\index{tests.conftest (module)@\spxentry{tests.conftest}\spxextra{module}}
The configuration for tests
The config is required by the pytest. The pytest will run this file at first.
\index{app() (in module tests.conftest)@\spxentry{app()}\spxextra{in module tests.conftest}}

\begin{fulllineitems}
\phantomsection\label{\detokenize{tests:tests.conftest.app}}\pysiglinewithargsret{\sphinxcode{\sphinxupquote{tests.conftest.}}\sphinxbfcode{\sphinxupquote{app}}}{}{}
Configure the flask app for testing

This is a pytest fixture

\begin{sphinxadmonition}{attention}{Attention:}
The the database is in progress, and not used. All files are saved in the folder defined in the ‘UPLOAD\_FOLDER’
in the app configurations
\end{sphinxadmonition}

\begin{sphinxadmonition}{warning}{Warning:}
The uploaded folder is relative to where the pytest is called. Calling pytest in other folder will result a mis-
placed uploaded folder. The uploaded folder should be kept under test directory
\end{sphinxadmonition}

\end{fulllineitems}

\index{client() (in module tests.conftest)@\spxentry{client()}\spxextra{in module tests.conftest}}

\begin{fulllineitems}
\phantomsection\label{\detokenize{tests:tests.conftest.client}}\pysiglinewithargsret{\sphinxcode{\sphinxupquote{tests.conftest.}}\sphinxbfcode{\sphinxupquote{client}}}{\emph{app}}{}
The test client for simulating requests
\begin{quote}\begin{description}
\item[{Returns}] \leavevmode
Return a test client

\end{description}\end{quote}

\end{fulllineitems}

\index{runner() (in module tests.conftest)@\spxentry{runner()}\spxextra{in module tests.conftest}}

\begin{fulllineitems}
\phantomsection\label{\detokenize{tests:tests.conftest.runner}}\pysiglinewithargsret{\sphinxcode{\sphinxupquote{tests.conftest.}}\sphinxbfcode{\sphinxupquote{runner}}}{\emph{app}}{}~
\begin{sphinxadmonition}{attention}{Attention:}
Not used now
\end{sphinxadmonition}

\end{fulllineitems}



\section{tests.test\_cloudmesh module}
\label{\detokenize{tests:module-tests.test_cloudmesh}}\label{\detokenize{tests:tests-test-cloudmesh-module}}\index{tests.test\_cloudmesh (module)@\spxentry{tests.test\_cloudmesh}\spxextra{module}}
Test the functions in {\hyperref[\detokenize{cloudmesh.analytics:module-cloudmesh.analytics.analytics}]{\sphinxcrossref{\sphinxcode{\sphinxupquote{cloudmesh.analytics.analytics}}}}}
\begin{quote}
\begin{description}
\item[{Tip:}] \leavevmode
Running the test under the cloudmesh-analytics directory

\sphinxcode{\sphinxupquote{{}`\textgreater{} ./cloudmesh-analytics\$ pytest{}`}}

\end{description}
\end{quote}
\index{TestFileOperations (class in tests.test\_cloudmesh)@\spxentry{TestFileOperations}\spxextra{class in tests.test\_cloudmesh}}

\begin{fulllineitems}
\phantomsection\label{\detokenize{tests:tests.test_cloudmesh.TestFileOperations}}\pysigline{\sphinxbfcode{\sphinxupquote{class }}\sphinxcode{\sphinxupquote{tests.test\_cloudmesh.}}\sphinxbfcode{\sphinxupquote{TestFileOperations}}}
Bases: \sphinxcode{\sphinxupquote{object}}

Test file operations

\begin{sphinxadmonition}{attention}{Attention:}\begin{enumerate}
\sphinxsetlistlabels{\arabic}{enumi}{enumii}{}{.}%
\item {} 
The function will be ran first and the files uploaded will be used for other tests

\end{enumerate}

2. The uploaded file is insulated and saved in the testing\_files directory as indicated in
{\hyperref[\detokenize{tests:module-tests.conftest}]{\sphinxcrossref{\sphinxcode{\sphinxupquote{tests.conftest}}}}}
\end{sphinxadmonition}
\index{post\_file() (tests.test\_cloudmesh.TestFileOperations method)@\spxentry{post\_file()}\spxextra{tests.test\_cloudmesh.TestFileOperations method}}

\begin{fulllineitems}
\phantomsection\label{\detokenize{tests:tests.test_cloudmesh.TestFileOperations.post_file}}\pysiglinewithargsret{\sphinxbfcode{\sphinxupquote{post\_file}}}{\emph{client}, \emph{path}, \emph{name}}{}
A helper function to make post request
\begin{quote}\begin{description}
\item[{Parameters}] \leavevmode\begin{itemize}
\item {} 
\sphinxstyleliteralstrong{\sphinxupquote{client}} \textendash{} The pytest fixture defined in {\hyperref[\detokenize{tests:module-tests.conftest}]{\sphinxcrossref{\sphinxcode{\sphinxupquote{tests.conftest}}}}}

\item {} 
\sphinxstyleliteralstrong{\sphinxupquote{path}} \textendash{} The rest api defined in the yaml file

\item {} 
\sphinxstyleliteralstrong{\sphinxupquote{name}} \textendash{} the file name to post

\end{itemize}

\item[{Returns}] \leavevmode
The data attribute of the flask response object

\end{description}\end{quote}

\end{fulllineitems}

\index{pytestmark (tests.test\_cloudmesh.TestFileOperations attribute)@\spxentry{pytestmark}\spxextra{tests.test\_cloudmesh.TestFileOperations attribute}}

\begin{fulllineitems}
\phantomsection\label{\detokenize{tests:tests.test_cloudmesh.TestFileOperations.pytestmark}}\pysigline{\sphinxbfcode{\sphinxupquote{pytestmark}}\sphinxbfcode{\sphinxupquote{ = {[}Mark(name='first', args=(), kwargs=\{\}){]}}}}
\end{fulllineitems}

\index{test\_format\_error() (tests.test\_cloudmesh.TestFileOperations method)@\spxentry{test\_format\_error()}\spxextra{tests.test\_cloudmesh.TestFileOperations method}}

\begin{fulllineitems}
\phantomsection\label{\detokenize{tests:tests.test_cloudmesh.TestFileOperations.test_format_error}}\pysiglinewithargsret{\sphinxbfcode{\sphinxupquote{test\_format\_error}}}{\emph{client}}{}
The upload will failed due to the txt file format. An error message will return
\begin{quote}\begin{description}
\item[{Parameters}] \leavevmode
\sphinxstyleliteralstrong{\sphinxupquote{client}} \textendash{} The pytest fixture defined in {\hyperref[\detokenize{tests:module-tests.conftest}]{\sphinxcrossref{\sphinxcode{\sphinxupquote{tests.conftest}}}}}

\item[{Returns}] \leavevmode
The data attribute of the flask response object, which is a binary string that includes a list of uploaded
file names

\end{description}\end{quote}

\end{fulllineitems}

\index{test\_read() (tests.test\_cloudmesh.TestFileOperations method)@\spxentry{test\_read()}\spxextra{tests.test\_cloudmesh.TestFileOperations method}}

\begin{fulllineitems}
\phantomsection\label{\detokenize{tests:tests.test_cloudmesh.TestFileOperations.test_read}}\pysiglinewithargsret{\sphinxbfcode{\sphinxupquote{test\_read}}}{\emph{client}}{}
Test read uploaded file using the rest api

\end{fulllineitems}

\index{test\_success\_upload() (tests.test\_cloudmesh.TestFileOperations method)@\spxentry{test\_success\_upload()}\spxextra{tests.test\_cloudmesh.TestFileOperations method}}

\begin{fulllineitems}
\phantomsection\label{\detokenize{tests:tests.test_cloudmesh.TestFileOperations.test_success_upload}}\pysiglinewithargsret{\sphinxbfcode{\sphinxupquote{test\_success\_upload}}}{\emph{client}}{}
Test upload. The file will be uploaded in to the current directory named files

The test sample will use a empty csv file called test upload
\begin{quote}\begin{description}
\item[{Parameters}] \leavevmode
\sphinxstyleliteralstrong{\sphinxupquote{client}} \textendash{} The pytest fixture defined in {\hyperref[\detokenize{tests:module-tests.conftest}]{\sphinxcrossref{\sphinxcode{\sphinxupquote{tests.conftest}}}}}

\item[{Returns}] \leavevmode
The data attribute of the flask response object, which is a binary string that includes a list of uploaded
file names

\end{description}\end{quote}

\end{fulllineitems}

\index{test\_success\_upload\_dabetes() (tests.test\_cloudmesh.TestFileOperations method)@\spxentry{test\_success\_upload\_dabetes()}\spxextra{tests.test\_cloudmesh.TestFileOperations method}}

\begin{fulllineitems}
\phantomsection\label{\detokenize{tests:tests.test_cloudmesh.TestFileOperations.test_success_upload_dabetes}}\pysiglinewithargsret{\sphinxbfcode{\sphinxupquote{test\_success\_upload\_dabetes}}}{\emph{client}}{}
Test upload. The file will be uploaded in to the current directory named files
The test sample will use a empty csv file called test upload
\begin{quote}\begin{description}
\item[{Parameters}] \leavevmode
\sphinxstyleliteralstrong{\sphinxupquote{client}} \textendash{} The pytest fixture defined in {\hyperref[\detokenize{tests:module-tests.conftest}]{\sphinxcrossref{\sphinxcode{\sphinxupquote{tests.conftest}}}}}

\item[{Returns}] \leavevmode
The data attribute of the flask response object, which is a binary string that includes a list of uploaded
file names

\end{description}\end{quote}

\end{fulllineitems}

\index{test\_success\_upload\_sample() (tests.test\_cloudmesh.TestFileOperations method)@\spxentry{test\_success\_upload\_sample()}\spxextra{tests.test\_cloudmesh.TestFileOperations method}}

\begin{fulllineitems}
\phantomsection\label{\detokenize{tests:tests.test_cloudmesh.TestFileOperations.test_success_upload_sample}}\pysiglinewithargsret{\sphinxbfcode{\sphinxupquote{test\_success\_upload\_sample}}}{\emph{client}}{}
Test upload. The file will be uploaded in to the current directory named files

The test sample will use a empty csv file called test upload
\begin{quote}\begin{description}
\item[{Parameters}] \leavevmode
\sphinxstyleliteralstrong{\sphinxupquote{client}} \textendash{} The pytest fixture defined in {\hyperref[\detokenize{tests:module-tests.conftest}]{\sphinxcrossref{\sphinxcode{\sphinxupquote{tests.conftest}}}}}

\item[{Returns}] \leavevmode
The data attribute of the flask response object, which is a binary string that includes a list of uploaded
file names

\end{description}\end{quote}

\end{fulllineitems}


\end{fulllineitems}

\index{TestKMeans (class in tests.test\_cloudmesh)@\spxentry{TestKMeans}\spxextra{class in tests.test\_cloudmesh}}

\begin{fulllineitems}
\phantomsection\label{\detokenize{tests:tests.test_cloudmesh.TestKMeans}}\pysigline{\sphinxbfcode{\sphinxupquote{class }}\sphinxcode{\sphinxupquote{tests.test\_cloudmesh.}}\sphinxbfcode{\sphinxupquote{TestKMeans}}}
Bases: \sphinxcode{\sphinxupquote{object}}
\index{test\_errors() (tests.test\_cloudmesh.TestKMeans method)@\spxentry{test\_errors()}\spxextra{tests.test\_cloudmesh.TestKMeans method}}

\begin{fulllineitems}
\phantomsection\label{\detokenize{tests:tests.test_cloudmesh.TestKMeans.test_errors}}\pysiglinewithargsret{\sphinxbfcode{\sphinxupquote{test\_errors}}}{\emph{client}}{}
Testing error arguments. The exception raised by the sci-kit learn will be returned in the error message.
The exception also raised by the filename doesn’t exist in app.config{[}‘UPLOAD\_FOLDER’{]}
:param client:
:return:

\end{fulllineitems}

\index{test\_kmeans\_fit() (tests.test\_cloudmesh.TestKMeans method)@\spxentry{test\_kmeans\_fit()}\spxextra{tests.test\_cloudmesh.TestKMeans method}}

\begin{fulllineitems}
\phantomsection\label{\detokenize{tests:tests.test_cloudmesh.TestKMeans.test_kmeans_fit}}\pysiglinewithargsret{\sphinxbfcode{\sphinxupquote{test\_kmeans\_fit}}}{\emph{client}}{}
\end{fulllineitems}


\end{fulllineitems}

\index{TestLinearRegression (class in tests.test\_cloudmesh)@\spxentry{TestLinearRegression}\spxextra{class in tests.test\_cloudmesh}}

\begin{fulllineitems}
\phantomsection\label{\detokenize{tests:tests.test_cloudmesh.TestLinearRegression}}\pysigline{\sphinxbfcode{\sphinxupquote{class }}\sphinxcode{\sphinxupquote{tests.test\_cloudmesh.}}\sphinxbfcode{\sphinxupquote{TestLinearRegression}}}
Bases: \sphinxcode{\sphinxupquote{object}}
\index{test\_errors() (tests.test\_cloudmesh.TestLinearRegression method)@\spxentry{test\_errors()}\spxextra{tests.test\_cloudmesh.TestLinearRegression method}}

\begin{fulllineitems}
\phantomsection\label{\detokenize{tests:tests.test_cloudmesh.TestLinearRegression.test_errors}}\pysiglinewithargsret{\sphinxbfcode{\sphinxupquote{test\_errors}}}{\emph{client}}{}
Testing error arguments. The exception raised by the sci-kit learn will be returned in the error message
\begin{quote}\begin{description}
\item[{Parameters}] \leavevmode
\sphinxstyleliteralstrong{\sphinxupquote{client}} \textendash{} The pytest fixture defined in {\hyperref[\detokenize{tests:module-tests.conftest}]{\sphinxcrossref{\sphinxcode{\sphinxupquote{tests.conftest}}}}}

\item[{Returns}] \leavevmode
The data attribute of the flask response object, which is a binary string that includes a list of uploaded
file names

\end{description}\end{quote}

\begin{sphinxadmonition}{note}{Note:}
The server will return the error message raised by the sci-kit learn
\end{sphinxadmonition}

\end{fulllineitems}

\index{test\_linear\_regression() (tests.test\_cloudmesh.TestLinearRegression method)@\spxentry{test\_linear\_regression()}\spxextra{tests.test\_cloudmesh.TestLinearRegression method}}

\begin{fulllineitems}
\phantomsection\label{\detokenize{tests:tests.test_cloudmesh.TestLinearRegression.test_linear_regression}}\pysiglinewithargsret{\sphinxbfcode{\sphinxupquote{test\_linear\_regression}}}{\emph{client}}{}
Testing error arguments. The exception raised by the sci-kit learn will be returned

The data is taken from the sci-kit learn built in samples.
\begin{quote}\begin{description}
\item[{Parameters}] \leavevmode
\sphinxstyleliteralstrong{\sphinxupquote{client}} \textendash{} The pytest fixture defined in {\hyperref[\detokenize{tests:module-tests.conftest}]{\sphinxcrossref{\sphinxcode{\sphinxupquote{tests.conftest}}}}}

\item[{Returns}] \leavevmode
The data attribute of the flask response object, which is a binary string that includes a list of uploaded
file names

\end{description}\end{quote}

\begin{sphinxadmonition}{warning}{Warning:}
Todo: The assertion may be false due to the floating number representaion in different word-size systems
\end{sphinxadmonition}

\end{fulllineitems}


\end{fulllineitems}

\index{teardown\_module() (in module tests.test\_cloudmesh)@\spxentry{teardown\_module()}\spxextra{in module tests.test\_cloudmesh}}

\begin{fulllineitems}
\phantomsection\label{\detokenize{tests:tests.test_cloudmesh.teardown_module}}\pysiglinewithargsret{\sphinxcode{\sphinxupquote{tests.test\_cloudmesh.}}\sphinxbfcode{\sphinxupquote{teardown\_module}}}{}{}
Teardown any state that was previously setup
Remove the test\_upload\_folder by the end of tests

\end{fulllineitems}

\index{test\_run\_pca() (in module tests.test\_cloudmesh)@\spxentry{test\_run\_pca()}\spxextra{in module tests.test\_cloudmesh}}

\begin{fulllineitems}
\phantomsection\label{\detokenize{tests:tests.test_cloudmesh.test_run_pca}}\pysiglinewithargsret{\sphinxcode{\sphinxupquote{tests.test\_cloudmesh.}}\sphinxbfcode{\sphinxupquote{test\_run\_pca}}}{\emph{client}}{}
\end{fulllineitems}



\section{Module contents}
\label{\detokenize{tests:module-tests}}\label{\detokenize{tests:module-contents}}\index{tests (module)@\spxentry{tests}\spxextra{module}}

\chapter{Requirements}
\label{\detokenize{progress:requirements}}\label{\detokenize{progress::doc}}

\section{Local Cloudmesh Command}
\label{\detokenize{progress:local-cloudmesh-command}}\begin{enumerate}
\sphinxsetlistlabels{\arabic}{enumi}{enumii}{}{.}%
\item {} 
using cloumesh command to start and stop the remote server?

\item {} 
how to deploy the project?

\end{enumerate}


\section{TODO}
\label{\detokenize{progress:todo}}\begin{enumerate}
\sphinxsetlistlabels{\arabic}{enumi}{enumii}{}{.}%
\item {} 
A cloudmesh client will communicate with the server

\item {} 
A cloudmesh client is required

\item {} 
12 functions are needed functionality

\item {} 
Exposing more service

\item {} 
Stop watch for testing

\item {} 
Set up docker and migrate to the cloud

\end{enumerate}

2019.11.01. 11:19:18
\begin{enumerate}
\sphinxsetlistlabels{\arabic}{enumi}{enumii}{}{.}%
\item {} 
use the command generate tool?

\item {} 
!{[}image-20191101112532678{]}(/Users/qiweiliu/Library/Application Support/typora-user-images/image-20191101112532678.png)

\item {} 
migrate the analytics folder to somewhere else
\begin{enumerate}
\sphinxsetlistlabels{\arabic}{enumii}{enumiii}{}{.}%
\item {} 
The folder structure

\end{enumerate}

\end{enumerate}

The goal is to generate open api and

\begin{sphinxVerbatim}[commandchars=\\\{\}]
\PYG{n}{cms} \PYG{n}{analytics} \PYG{n}{run} \PYG{n}{linear}\PYG{o}{\PYGZhy{}}\PYG{n}{regression} \PYG{o}{\PYGZhy{}}\PYG{n}{data} \PYG{o}{=} \PYG{n+nb}{file}\PYG{o}{.}\PYG{n}{csv} \PYG{o}{\PYGZhy{}}\PYG{n}{intercept}\PYG{o}{=}\PYG{n}{true}
\end{sphinxVerbatim}
\begin{enumerate}
\sphinxsetlistlabels{\arabic}{enumi}{enumii}{}{.}%
\item {} 
Match the function name and parameters

\item {} 
Call the function matching the user input
\begin{enumerate}
\sphinxsetlistlabels{\arabic}{enumii}{enumiii}{}{.}%
\item {} 
Return error raised by sklearn

\end{enumerate}

\end{enumerate}


\chapter{Indices and tables}
\label{\detokenize{index:indices-and-tables}}\begin{itemize}
\item {} 
\DUrole{xref,std,std-ref}{genindex}

\item {} 
\DUrole{xref,std,std-ref}{modindex}

\item {} 
\DUrole{xref,std,std-ref}{search}

\end{itemize}


\renewcommand{\indexname}{Python Module Index}
\begin{sphinxtheindex}
\let\bigletter\sphinxstyleindexlettergroup
\bigletter{c}
\item\relax\sphinxstyleindexentry{cloudmesh}\sphinxstyleindexpageref{cloudmesh:\detokenize{module-cloudmesh}}
\item\relax\sphinxstyleindexentry{cloudmesh.analytics}\sphinxstyleindexpageref{cloudmesh.analytics:\detokenize{module-cloudmesh.analytics}}
\item\relax\sphinxstyleindexentry{cloudmesh.analytics.analytics}\sphinxstyleindexpageref{cloudmesh.analytics:\detokenize{module-cloudmesh.analytics.analytics}}
\item\relax\sphinxstyleindexentry{cloudmesh.analytics.api}\sphinxstyleindexpageref{cloudmesh.analytics.api:\detokenize{module-cloudmesh.analytics.api}}
\item\relax\sphinxstyleindexentry{cloudmesh.analytics.api.manager}\sphinxstyleindexpageref{cloudmesh.analytics.api:\detokenize{module-cloudmesh.analytics.api.manager}}
\item\relax\sphinxstyleindexentry{cloudmesh.analytics.command}\sphinxstyleindexpageref{cloudmesh.analytics.command:\detokenize{module-cloudmesh.analytics.command}}
\item\relax\sphinxstyleindexentry{cloudmesh.analytics.command.analytics}\sphinxstyleindexpageref{cloudmesh.analytics.command:\detokenize{module-cloudmesh.analytics.command.analytics}}
\item\relax\sphinxstyleindexentry{cloudmesh.analytics.file}\sphinxstyleindexpageref{cloudmesh.analytics:\detokenize{module-cloudmesh.analytics.file}}
\item\relax\sphinxstyleindexentry{cloudmesh.analytics.file\_helpers}\sphinxstyleindexpageref{cloudmesh.analytics:\detokenize{module-cloudmesh.analytics.file_helpers}}
\item\relax\sphinxstyleindexentry{cloudmesh.analytics.server}\sphinxstyleindexpageref{cloudmesh.analytics.server:\detokenize{module-cloudmesh.analytics.server}}
\item\relax\sphinxstyleindexentry{cloudmesh.analytics.server.db}\sphinxstyleindexpageref{cloudmesh.analytics.server:\detokenize{module-cloudmesh.analytics.server.db}}
\item\relax\sphinxstyleindexentry{cloudmesh.analytics.server.server}\sphinxstyleindexpageref{cloudmesh.analytics.server:\detokenize{module-cloudmesh.analytics.server.server}}
\indexspace
\bigletter{t}
\item\relax\sphinxstyleindexentry{tests}\sphinxstyleindexpageref{tests:\detokenize{module-tests}}
\item\relax\sphinxstyleindexentry{tests.conftest}\sphinxstyleindexpageref{tests:\detokenize{module-tests.conftest}}
\item\relax\sphinxstyleindexentry{tests.test\_cloudmesh}\sphinxstyleindexpageref{tests:\detokenize{module-tests.test_cloudmesh}}
\end{sphinxtheindex}

\renewcommand{\indexname}{Index}
\printindex
\end{document}